\begin{abstract}

The increasing availability of medical imaging exams, such as magnetic resonance imaging, generates a large volume of data, making its analysis complex and challenging. In this scenario, advanced computational approaches can optimize the interpretation of these images and assist in the early diagnosis of cardiovascular diseases. This work aims to unify contemporary approaches in the evaluation of cardiomyopathy. With the support of radiomic analysis, which extracts information from the statistical and texture characteristics of a medical image, and features derived from a classical neural network for computer vision, such as ResNet50, promising results can be obtained. The results confirm that the combination of information from various domains regarding a given patient, when integrated, can lead to more interesting outcomes compared to analyzing data in isolation. This study aims to apply the aforementioned approaches, based on previous literature, in an innovative application for cardiomyopathy testing, adapting and proposing a more robust architecture to achieve better results.

\keywords{Radiomics, Attention Mechanism, Transformers, Cardiomyopathy}
\end{abstract}