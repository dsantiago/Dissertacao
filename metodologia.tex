\chapter{Metodologia} 
\label{chap:metodologia}

O objetivo deste trabalho é propor e investigar uma arquitetura de aprendizado profundo que unifica \textit{features} de radiômica e \textit{features} profundas. Inicialmente, é empregado diversas técnicas de aprendizado de máquina para extrair \textit{features} manuais de imagens de RM, abrangendo textura, forma, escala de cinza, etc. Posteriormente, uma rede \textit{ResNet50} pré-treinada é utilizada para extrair \textit{features} profundas que encapsulam informações semânticas de alto nível e de representação das imagens de RM. Estas \textit{features} são então fundidas em um vetor de \textit{features} unificado. Para aprimorar a acurácia e a robustez, um módulo de \textit{self attention} foi desenvolvido. Utilizando o mecanismo de \textit{self attention}, este módulo otimiza e pondera o vetor de \textit{features} fundidas de forma eficaz.

%---------------------------------------------------------
\section{Conjunto de Dados}
As bases de imagens utilizadas neste trabalho não possuem acesso público. O acesso
a base ocorreu pela parceria existente entre o Centro Universitário FEI e o Instituto do Coração do Hospital das Clínicas da FMUSP (InCor).

%---------------------------------------------------------
\section{Métodos}
\label{sec:cap4_metodos}

%---------------------------------------------------------
\subsection{Features Radiômicas}
\label{subsec:cap4_features_radiomicas}

Foi extraído 72 \textit{features} radiômicas de fase diastólica, representada por um conjunto de fatias variando entre 6 e 18 \textit{frames}, de cada paciente usando matriz de coocorrência de níveis de cinza (GLCM) e estatísticas baseadas em histograma. Foi aplicado o filtro \textit{Laplace of Gaussian} (LoG) com cinco diferentes valores em cada parte para suavizar as imagens e realçar as bordas. Foi calculado \textit{features} GLCM como contraste, entropia, correlação, homogeneidade e energia para cada filtro LoG. Também é calculado \textit{features} de intensidade como média, variância, média dos percentis (10 e 90), desvio robusto da média absoluta, curtose e assimetria usando estatísticas de primeira ordem. Foram obtidos 78 \textit{features} radiômicas para cada paciente dentro da quantidade de fatias extraídas na fase diastólica.

%---------------------------------------------------------
\subsection{Features Profundas}
\label{subsec:cap4_features_profundas}
 
Para extrair características profundas das imagens de RM, é utilizada a arquitetura pré-treinada de um \textit{ResNet50} sem sua última camada totalmente conectada, treinada no conjunto de dados ImageNet. Estudos anteriores demonstraram que o pré-treinamento com ImageNet pode melhorar o desempenho de tarefas de classificação de imagens médicas. ResNet50 é um modelo de rede neural convolucional profunda com 50 camadas que compreende muitos blocos residuais. Cada bloco contém módulos de convolução e uma conexão de salto que transfere a informação do bloco anterior para o próximo bloco. A conexão de salto ajuda a reter a informação semântica mais básica aprendida nas camadas anteriores, que de outra forma se tornaria abstrata devido à conexão de longa cadeia. A conexão de salto também evita o desaparecimento do gradiente nas camadas mais profundas ao fornecer um caminho alternativo para a retropropagação. A informação da conexão de salto é adicionada à informação calculada em cada bloco \cite{aiSelfAttentionBasedFusion2023}. Ao todo são 2048 \textit{features} coletadas da saída deste modelo.

%---------------------------------------------------------
\subsection{Unificando as Features}
\label{subsec:cap4_unificando_features}

Um vez em posse das features radiômicas, é aplicado um F-Test tanto às 78 \textit{features} radiômicas quanto às 2048 \textit{features} profundas, reduzindo cada um dos vetores ao espaço de 64 \textit{features}. No método de fusão convencional, é simplesmente concatenado os dois vetores de \textit{features} como na Eq. \ref{eq:concat}, onde \textit{Concat} simplesmente concatena os dois vetores. Unificando ambos os vetores obtemos um vetor de 128 \textit{features} o qual passará pelo mecanismo de \textit{self-attention}.

\begin{equation}
F_{hd} = \textit{Concat}(F_h, F_d)
\label{eq:concat}
\end{equation}

%---------------------------------------------------------
\subsection{Módulo de SelfAttention}
\label{subsec:cap4_mod_self_attention}

Neste trabalho é empregado o mecanismo de \textit{self-attention} para aprender a importância de cada \textit{feature} e capturar suas dependências de longo alcance. Como ilustrado na 


Figura 1, o módulo de fusão de autoatenção é utilizado para mapear uma consulta (), chave () e valor () para um valor de atenção. Pegamos as 24 características concatenadas  como tokens e projetamos cada característica em três matrizes aprendíveis: matriz chave , matriz consulta  e matriz valor  por produto escalar com as matrizes ,  e . Os , ,  são denotados como , ,  respectivamente, onde ,  e  representam as matrizes de transformação linear para , , . O módulo de fusão baseado em autoatenção é definido da seguinte forma:

%---------------------------------------------------------

% \subsection{Unificando as Features}
% \label{subsec:cap4_unificando_features}

% Um vez em posse das features radiômicas, é aplicado um F-Test e reduz-se cada um dos vetores à 64 \textit{features}. Unificando ambos os vetores obtemos um vetor de \textit{features} de tamanho 128 o qual passará pelo mecanismo de \textit{attention}.

%---------------------------------------------------------
\section{Considerações Finais do Capítulo}
\label{sec:cap4_consideracoes_finais}

\lipsum[1-4]