\chapter{Metodologia} 
\label{chap:metodologia}

O objetivo deste trabalho é propor e investigar uma arquitetura de aprendizado profundo que unifica \textit{features} de radiômica e \textit{features} profundas. Inicialmente, é empregado diversas técnicas de aprendizado de máquina para extrair \textit{features} manuais de imagens de RM, abrangendo textura, forma, escala de cinza, etc. Posteriormente, uma rede \textit{ResNet50} pré-treinada é utilizada para extrair \textit{features} profundas que encapsulam informações semânticas de alto nível e de representação das imagens de RM. Estas \textit{features} são então fundidas em um vetor de \textit{features} unificado. Para aprimorar a acurácia e a robustez, um módulo de \textit{self attention} foi desenvolvido. Utilizando o mecanismo de \textit{self attention}, este módulo otimiza e pondera o vetor de \textit{features} fundidas de forma eficaz.

\section{Conjunto de Dados}
As bases de imagens utilizadas neste trabalho não possuem acesso público. O acesso
a base ocorreu pela parceria existente entre o Centro Universitário FEI e o Instituto do Coração do Hospital das Clínicas da FMUSP (InCor).



\section{Métodos}
\label{sec:cap4_metodos}

\subsection{Features Radiômicas}
\label{subsec:cap4_features_radiomicas}

Foi extraído 72 \textit{features} radiômicas de fase diastólica, representada por um conjunto de fatias variando entre 6 e 18 \textit{frames}, de cada paciente usando matriz de coocorrência de níveis de cinza (GLCM) e estatísticas baseadas em histograma. 

Foi aplicado o filtro \textit{Laplace of Gaussian} (LoG) com cinco diferentes valores em cada parte para suavizar as imagens e realçar as bordas. Foi calculado \textit{features} GLCM como contraste, entropia, correlação, homogeneidade e energia de quatro direções (1°, 45°, 90° e 135°) para cada filtro LoG. Também é calculado \textit{features} de intensidade como média, variância, média dos percentis (10 e 90), desvio robusto da média absoluta, curtose e assimetria usando estatísticas de primeira ordem. Foram obtidos 78 \textit{features} radiômicas para cada paciente dentro da quantidade de fatias existentes na fase diastólica.


\section{Considerações Finais do Capítulo}
\label{sec:cap4_consideracoes_finais}

\lipsum[1-4]