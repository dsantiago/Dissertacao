\begin{resumo}
% \lipsum[1-2]
Este trabalho consiste em unificar abordagens contemporâneas na avaliação da cardiomiopatia. Como apoio da análise radiômica, a qual extrai-se informações das características estatísticas e de textura de uma imagem médica e de \textit{features} oriundas de uma rede neural clássica para visão computacional, a \textit{ResNet50}, foi possível obter resultados promissores. Os resultados certificam que a união de informações de diversos âmbitos, a cerca de um dado paciente, quando aliadas, podem culminar em resultados mais interessados comparados quando avaliados os dados de forma isolada. O presente trabalho visa, usar as abordagens citadas, baseado em literaturas prévias, efetuando uma aplicação inédita para o teste de cardiomiopatia, adaptando e propondo uma arquitetura mais robusta de forma obter bons resultados.

\palavraschave{Radiomics, Mecanismo de Atenção, Transformers, Cardiomiopatia}
 
\end{resumo}