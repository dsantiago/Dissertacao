\chapter{Introdução}
\label{chap:intro}

A tecnologia tem perpetrado cada vez mais diversas áreas do saber, trazendo diversos benefícios e praticidades a vida contemporânea. Destas diferentes áreas, uma das que mais se beneficiaram com a tecnologia e a inovação foi a área médica. Desde meados dos anos 2000, a quantidade de dados disponível vêm crescendo exponencialmente se alcançando uma projeção de centenas de milhares de \gls{EB} para os anos de 2020 em diante \cite{gantzDIGITALUNIVERSE2020}.

Dada a enorme quantidade de dados gerados, a medicina tem se apoiado na utilização de dados clínicos textuais e principalmente imagens médicas, para a composição de seus diagnósticos. A análise de imagens médicas radiômicas é uma área em crescimento na medicina que se concentra na extração e análise de informações quantitativas e qualitativas destas imagens médicas, como \gls{TC}, \gls{RM} e imagens de ultrassom. Tanto a \gls{TC} quanto a \gls{RM} são exemplos de exames que podem gerar objetos tridimensionais (3D) de estruturas específicas do corpo humano a partir de dezenas de fatias que são imagens em um dado instante \cite{book:1355375}. No entanto, apesar do progresso das técnicas de imagem cardíaca, o coração permanece sendo um órgão desafiador para se estudar. A \gls{IA} emergiu como uma das principais inovações no campo da imagem diagnóstica, com um impacto dramático na \gls{RMC}. A \gls{IA} estará cada vez mais presente no mundo médico, com forte potencial para maior eficiência e precisão diagnóstica \cite{argentieroApplicationsArtificialIntelligence2022}.

A IA, composta por redes neurais profundas, também chamada de \gls{DL} dentro do espectro da visão computacional, pode ser utilizadas como classificadores, detectores de objetos, segmentadores, etc. Estas redes, com suas múltiplas camadas profundas, geram \textit{features} discriminantes após otimizadas a cerca de um conjunto de dados. Outro modelo de rede que vem tendo um uso cada vez maior e se encontra em diversas arquiteturas consagradas atuais, são as redes \textit{transformers}. As redes \textit{transformers} ficaram populares por serem comumente usadas em redes generativas auto-regressivas para geração sintética de texto, também conhecidas como \gls{LLM}, tendo como como seu exemplo mais conhecido o \textit{ChatGPT}. Os \textit{transformers} são arquiteturas que possuem como ponto forte principalmente sua capacidade de paralelismo e seu mecanismo de \textit{self-attention} que permite que o modelo se concentre nas partes relevantes dos dados de entrada, aprimorando a compreensão do contexto e das relações dentro dos dados.

Outra técnica também muito utilizada na análise de imagens médicas é a análise de textura que vem sendo usada por várias décadas em diversos domínios da medicina. Em oncologia, a análise de textura de imagens de \gls{TC} mostrou correlação com a biologia tumoral subjacente ao diferenciar características histológicas diferentes e mutações genéticas específicas. Em malignidades estabelecidas, a análise de textura se relaciona com a histologia do tumor em muitos tumores sólidos comuns (pulmão, colorretal, esofágico, mama), ela se correlaciona com mutações genéticas específicas e pode acompanhar respostas terapêuticas. Fora da oncologia, mudanças não malignas em órgãos podem ser detectadas (por exemplo, cirrose hepática e pneumonite intersticial usual). Dentro da imagem cardíaca, a análise de textura aplicada à \gls{RMC} tem sido usada para avaliar o risco de arritmia pós-infarto do miocárdio, o uso da RMC em imagens sem contraste e com realce tardio de gadolínio em pacientes com \gls{CH} para prever o resultado é uma área de interesse particular atualmente \cite{schofieldTextureAnalysisCardiovascular2019a}.

A área da análise radiômica consiste no processo em obter dados qualitativos destas imagens de radiografias combinadas, e por vezes, a extração também da textura destas. A análise radiômica pode até ser usada para avaliar a resposta ao tratamento ou para transmitir certas características prognósticas. Dentre as doenças do coração, a \gls{CH} é desenvolvida prioritariamente em jovens e pessoas de meia idade, sendo a mais comum das cardiomiopatias . Mesmo que a \gls{CH} se apresente de forma assintomática em parte dos casos, esta pode culminar em morte súbita cardíaca, insuficiência cardíaca, acidente vascular cerebral e arritmia . A análise radiômica auxilia com o diagnóstico prévio afim de compreender e atuar nos casos de \gls{CH} que demonstrem risco ao paciente \cite{kwonComparisonMortalityCause2022}.

% Algumas de suas características são: extração de características, quantificação e padronização, possibilidade de correlação com dados clínicos. As aplicações da análise radiômica consiste em diversas aplicações clínicas, incluindo diagnóstico auxiliado por computador, predição de resposta ao tratamento, estratificação de pacientes, identificação de biomarcadores e personalização de tratamentos.

Este trabalho tem como ideia principal unir tanto as \textit{features} radiômicas quanto as \textit{features} profundas, assim chamadas por serem extraídas de um modelo de \gls{DL}, e propor um modelo de redes neurais que se utiliza de \textit{self-attention}, para analisar e prever a \gls{CH}. O uso destas duas abordagens aliadas na extração de \textit{features} a cerca da doença, tem o potencial de obter resultados melhores resultados preditivos se comparados as técnicas isoladas. Neste trabalho é proposto uma estratégia de fusão de múltiplas fontes, que combina as \textit{features} manuais extraídas da análise de textura radiômica e as \textit{features} profundas, para melhorar a expressividade e a habilidade de generalização do modelo.

\newpage

% \section{Motivação}
%---------------------------------------------------------
\section{Objetivo}
\label{sec:cap1_objetivo}

%---------------------------------------------------------
\subsection{Objetivos Específicos}

Como objetivo desta dissertação, espera-se obter os seguintes resultados:

\begin{enumerate}

\item Execução de modelo, que utilizas as \textit{features} radiômicas e
\textit{deep features} e coleta de resultados em conjunto de dados públicos.

\item Avaliação dos resultados iniciais

\item Proposta de novas abordagens tanto no pré-processamento quanto na arquitetura, mediante coleta dos resultados.

\item Iteração e análise dos resultados obtidos, sua documentação e comparação.
\end{enumerate}

%---------------------------------------------------------
\section{Estrutura do trabalho}
\label{sec:cap1_estrutura_trabalho}

Esta proposta de dissertação está organizada em seis capítulos, a saber:
O \textbf{Capítulo \ref{chap:intro}} apresenta a introdução, motivação, objetivo e contribuições desejadas. O \textbf{Capítulo \ref{chap:fundamentacao_teorica}} fará um aprofundamento teórico no tema de pesquisa estudando as técnicas consideradas primordiais deste trabalho e/ou que são utilizadas no desenvolvimento do experimento prático. No \textbf{Capítulo \ref{chap:trab_relacionados}} é estudado  os trabalhos recentes, não superior a cinco anos passados, de outros autores que estão de alguma forma relacionados com tema, sendo estes trabalhos relacionados oriundos de uma minuciosa revisão sistemática da literatura. No \textbf{Capítulo \ref{chap:metodologia}} é apresentada uma proposta de metodologia para a realização do experimento prático.

O \textbf{Capítulo \ref{chap:proposta_experimental}} é uma proposta experimental que contempla os dados utilizados, informações do seu pré-processamento, hiperparâmetros utilizados e demais informações assim como o cronograma do presente projeto de pesquisa. No \textbf{Capítulo \ref{chap:resultados_discussao}}, são apresentados os resultados parciais da proposta experimental como também discutidos e argumentados os resultados parciais. Finalmente, o \textbf{Capítulo \ref{chap:conclusao}} apresenta conclusão do presente projeto de pesquisa.
