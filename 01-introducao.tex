\chapter{INTRODUÇÃO}
\label{chap:intro}

A tecnologia está cada vez mais presente nas diversas áreas do conhecimento, trazendo uma infinidade de benefícios e facilitando o cotidiano contemporâneo. Entre os diversos campos impactados, a área médica destaca-se como uma das que mais se beneficiaram da inovação tecnológica. Desde meados dos anos 2000, a quantidade de dados gerados na medicina tem crescido exponencialmente, atingindo projeções de milhares de \gls{EB} a partir de 2020 \cite{gantzDIGITALUNIVERSE2020}. Esse cenário evidencia a importância de ferramentas que possam processar e analisar eficientemente grandes volumes de informações.

% A tecnologia está presente cada vez mais nas diversas áreas do saber, trazendo diversos benefícios e praticidades a vida contemporânea. Dentre as diferentes áreas do saber, uma das que mais se beneficiaram com a tecnologia e a inovação foi a área médica. Desde meados dos anos 2000, a quantidade de dados disponível tem crescendo exponencialmente alcançando projeções de milhares de \gls{EB} a partir dos anos de 2020 em diante .

Exames de imagem, como a \gls{TC} e a \gls{RMC}, tornaram-se essenciais na medicina moderna. Esses exames não apenas oferecem uma representação tridimensional detalhada de estruturas do corpo humano, mas também produzem dados que podem ser analisados de forma quantitativa. A emergência da \gls{IA} trouxe avanços significativos à análise de imagens diagnósticas, proporcionando maior eficiência e precisão nos diagnósticos médicos \cite{argentieroApplicationsArtificialIntelligence2022}.

% A medicina vem gerando cada vez mais dados, sejam textuais, clínicos ou de imagem, e se utilizado deles para comporem seus diagnósticos. Tanto a \gls{TC} quanto a \gls{RMC} são exemplos de exames que podem gerar objetos tridimensionais (3D) de estruturas específicas do corpo humano a partir de dezenas de fatias que representam de forma bidimensional (2D), determinada estrutura do corpo humano dado um instante de tempo $t$. A \gls{IA} emergiu como uma das principais inovações no campo da imagem diagnóstica, com um impacto substancial em toda área médica inclusive com a análise de \gls{RMC}. A \gls{IA} estará cada vez mais presente no mundo médico, com forte potencial para maior eficiência e precisão diagnóstica \cite{argentieroApplicationsArtificialIntelligence2022}.

As redes neurais profundas, um dos principais pilares da \gls{IA}, têm demonstrado alto desempenho em tarefas de visão computacional, como classificação de imagens, detecção de objetos e segmentação. Essas redes conseguem aprender características discriminantes uma vez otimizadas a cerca do conjunto de dados em que foi treinada. Além disso, arquiteturas como o \textit{transformers} ficaram populares por serem comumente usadas em redes generativas auto-regressivas para geração sintética de texto, também conhecidas como \gls{LLM}, tendo como como seu exemplo mais conhecido o \textit{ChatGPT}. Os \textit{transformers} são arquiteturas que destacam-se pela capacidade de paralelismo e pelo uso de mecanismos de autoatenção, que permitem ao modelo focar nas partes mais relevantes dos dados de entrada \cite{russell2020artificial}.

Paralelamente, técnicas de processamento de imagem como a análise de textura já vem sendo utilizada por várias décadas em diversos domínios da medicina. A análise radiômica emergiu como uma ferramenta poderosa na extração de informações quantitativas de imagens médicas, capturando padrões que muitas vezes passam despercebidos ao olho humano. Essa abordagem tem mostrado potencial em diversas áreas, como oncologia e cardiologia. No âmbito da \gls{CMH}, por exemplo, estudos têm demonstrado que a análise radiômica pode prever riscos como arritmia ou morte súbita \cite{schofieldTextureAnalysisCardiovascular2019a}.


\textcolor{red}{< < ------------------------------------- REMOVER ??? ------------------------------------- }

Em oncologia, a análise de textura de imagens de \gls{TC} mostrou correlação com a biologia tumoral subjacente, ao analisar características histológicas diferentes e mutações genéticas específicas. A biologia tumoral subjacente representa mecanismos biológicos fundamentais que estão na base do desenvolvimento, crescimento e progressão de tumores. Em problemas de tumores malignos já bem estabelecidos, a análise de textura se relaciona com a histologia nos exemplos sólidos comuns (pulmão, colorretal, esofágico, mama), se correlaciona com mutações genéticas específicas e pode acompanhar respostas terapêuticas. Fora da oncologia, mudanças não malignas em órgãos podem ser detectadas, como por exemplo cirrose hepática e pneumonite intersticial. No domínio da análise de imagens cardíacas, a análise de textura aplicada à \gls{RMC} tem avaliado o risco de arritmia pós-infarto do miocárdio. O uso de análise de textura para \gls{RMC} em imagens sem contraste e com realce tardio de gadolínio em pacientes com \gls{CMH} para prever o resultado do exame é uma área particular de interesse \cite{schofieldTextureAnalysisCardiovascular2019a}.

\textcolor{red}{--------------------------------------------------------------------------------------- > >}


A \gls{CMH} é uma das cardiomiopatias mais comuns, frequentemente diagnosticada em jovens e indivíduos de meia-idade. Embora em muitos casos seja assintomática, a doença pode levar a condições graves, como insuficiência cardíaca e acidente vascular cerebral. Isso torna o diagnóstico precoce essencial para prevenir desfechos adversos \cite{kwonComparisonMortalityCause2022}. Neste sentido a análise radiômica que consiste em extrair dados qualitativos de imagens radiográficas combinadas, incluindo, em muitos casos, a análise da textura dessas imagens \cite{lambinRadiomicsExtractingMore2012}, é usada para avaliar a resposta ao tratamento ou para transmitir certas características prognósticas. A análise radiômica auxilia com o diagnóstico prévio afim de compreender e atuar nos casos de \gls{CMH} que demonstrem risco ao paciente.

%-----------------------------------------

Neste cenário, combinar técnicas de \gls{IA} e análise radiômica representa uma estratégia promissora para a detecção de CMH e outras condições cardíacas. Estudos recentes, como o de \citeonline{aiSelfAttentionBasedFusion2023}, demonstraram que a integração de características profundas e radiômicas pode melhorar significativamente o desempenho preditivo de modelos diagnósticos. Esses modelos baseiam-se em mecanismos de autoatenção para identificar padrões relevantes em dados concatenados, alcançando acurácia de até $82,35\%$ e AUC de $0,74$.


% A análise radiômica em conjunto com técnicas de \gls{IA} tem um grande potencial e há pesquisas que já tentam unificar os dois campos. O trabalho de \citeonline{aiSelfAttentionBasedFusion2023}, propôs a união de características radiômicas e características profundas, assim chamadas por serem extraídas de um modelo de \gls{AP}. O modelo proposto utiliza mecanismo de autoatenção para analisar e prever \gls{CMH}, tendo como entrada as características concatenadas. O uso destas duas abordagens aliadas na extração de características a cerca da doença, tem o potencial de obter melhores resultados preditivos se comparados as técnicas isoladas. Os resultado deste trabalho obtiveram acurácia de $\Acc\%$ e AUC de $\Auc$.

Assim, o presente trabalho propõe a implementação e validação de uma estratégia de fusão que combina características radiômicas e profundas, com o uso de mecanismos de autoatenção, para melhorar a classificação de \gls{CMH}. Além de avançar o estado da arte em diagnósticos médicos, esta pesquisa busca contribuir para a adoção de soluções mais eficazes e acessíveis no apoio à decisão clínica.

\newpage
% \clearpage

%---------------------------------------------------------
\section{OBJETIVOS}
\label{sec:cap1_objetivo}


Os objetivos deste trabalho foram estruturados com a finalidade de propor e validar uma abordagem inovadora para a classificação de cardiomiopatias utilizando técnicas de inteligência artificial e análise radiômica. Estes objetivos foram definidos para responder às necessidades clínicas e avançar o estado da arte na área de diagnóstico médico. A seguir, são detalhados os objetivos gerais e específicos:

%---------------------------------------------------------
\subsection{Objetivos Gerais}
\label{subsec:cap1_objetivo_geral}

\begin{enumerate}
\item Desenvolver, implementar e validar um modelo de classificação de cardiomiopatias, com ênfase na \gls{CMH}, utilizando a integração de características radiômicas e profundas mediadas por mecanismos de autoatenção.
\end{enumerate}

%---------------------------------------------------------
\subsection{Objetivos Específicos}
\label{subsec:cap1_objetivo_especific}

\begin{enumerate}
\item Identificar e extrair características radiômicas de imagens de \gls{RMC}, capturando padrões texturais e estatísticos relevantes para a classificação das cardiomiopatias.

\item Projetar um pipeline de análise que combine eficientemente características radiômicas e profundas, garantindo a fusão informativa para modelos de aprendizado profundo.

\item Implementar um modelo baseado em redes neurais profundas que utilize mecanismos de atenção para priorizar regiões relevantes nas imagens, melhorando a acurácia e a interpretabilidade do modelo.

\item Validar a eficácia do modelo proposto utilizando métricas padrão, como acurácia, precisão, revocação, F1-score e AUC, em um conjunto de dados públicos e diversificado.

\item Comparar o desempenho do modelo proposto com técnicas existentes na literatura, identificando avanços e limitações em relação às abordagens tradicionais.

\item Explorar o potencial de generalização do modelo em outras condições clínicas, em específico a \gls{CMH}, sugerindo aplicações futuras para diagnóstico assistido por inteligência artificial.
\end{enumerate}

% \begin{enumerate}
% \item Implementar e processar modelo que unifica informações radiômicas e profundas, aplicado à análise de câncer do pulmão, em conjunto de dados público e privados de \gls{CMH}.

% \item Avaliação dos resultados iniciais.

% \item Adicionar novas estrutura ao modelo base e treinar com os dados públicos e privados.

% \item Iteração e análise dos resultados obtidos, sua documentação e comparação.

% % \item Implementação do modelo, que utiliza características radiômicas e
% % características profundas.

% % \item Coleta de resultados em conjunto de dados públicos, como linha de base da análise.

% % \item Avaliação dos resultados iniciais.

% % \item Proposta de novas abordagens tanto no pré-processamento quanto na arquitetura, mediante coleta dos resultados.

% % \item Iteração e análise dos resultados obtidos, sua documentação e comparação.
% \end{enumerate}

%---------------------------------------------------------
\section{Estrutura do trabalho}
\label{sec:cap1_estrutura_trabalho}

Este trabalho está organizado em sete capítulos, cada um projetado para abordar diferentes aspectos da pesquisa e sua implementação. A seguir, apresenta-se a estrutura detalhada desta pesquisa:

O \textbf{Capítulo \ref{chap:intro}} apresenta o contexto geral do trabalho, destacando a relevância do uso de inteligência artificial e análise radiômica no diagnóstico de cardiomiopatias. São expostos os objetivos gerais e específicos da pesquisa, bem como a motivação para o desenvolvimento deste estudo. O \textbf{Capítulo \ref{chap:fundamentacao_teorica}} discute os conceitos teóricos que embasam o trabalho, incluindo princípios de redes neurais profundas, análise radiômica, mecanismos de autoatenção e sua aplicação em imagens médicas. Este capítulo também revisa estudos relacionados que contribuíram para o estado da arte na área. No \textbf{Capítulo \ref{chap:trab_relacionados}} é estudado  os trabalhos recentes, não superior a cinco anos passados, de outros autores que estão de alguma forma relacionados com tema, sendo estes trabalhos relacionados oriundos de uma minuciosa revisão sistemática da literatura. O \textbf{Capítulo \ref{chap:metodologia}} detalha o método proposto para a classificação de cardiomiopatias, incluindo a descrição do conjunto de dados utilizado, o processo de extração de características radiômicas, o desenvolvimento do modelo baseado em redes neurais profundas e as etapas de validação e experimentação. O \textbf{Capítulo \ref{chap:proposta_experimental}} faz a proposição experimental que contempla os dados utilizados, informações do seu pré-processamento, hiperparâmetros planejados e demais informações a cerca do experimentos realizados.  No \textbf{Capítulo \ref{chap:resultados_experimentais}} é apresentado os resultados obtidos com a implementação do modelo proposto, acompanhados de análises quantitativas e qualitativas. Discute-se o desempenho do modelo em comparação com outras abordagens e suas implicações técnicas. Finalmente, o \textbf{Capítulo \ref{chap:cap7_conclusao}} sintetiza os principais achados da pesquisa, destacando as contribuições do trabalho para o estado da arte e suas potenciais aplicações. Além disso, são sugeridas direções para estudos futuros que possam expandir e aprimorar as técnicas apresentadas.


% Esta proposta está organizada em seis capítulos, a saber:
% O \textbf{Capítulo \ref{chap:intro}} apresenta a introdução, objetivo e contribuições desejadas. O \textbf{Capítulo \ref{chap:fundamentacao_teorica}} faz um aprofundamento teórico no tema de pesquisa estudando as técnicas consideradas primordiais deste trabalho e/ou que são utilizadas no desenvolvimento do experimento prático. No \textbf{Capítulo \ref{chap:trab_relacionados}} é estudado  os trabalhos recentes, não superior a cinco anos passados, de outros autores que estão de alguma forma relacionados com tema, sendo estes trabalhos relacionados oriundos de uma minuciosa revisão sistemática da literatura. No \textbf{Capítulo \ref{chap:metodologia}} é apresentada uma proposta de metodologia para a realização do experimento prático. O \textbf{Capítulo \ref{chap:proposta_experimental}} é uma proposta experimental que contempla os dados utilizados, informações do seu pré-processamento, hiperparâmetros planejados e demais informações assim como o cronograma do presente projeto de pesquisa. No \textbf{Capítulo \ref{chap:resultados_experimentais}}, são apresentados os resultados parciais da prova de conceitos, considerando a arquitetura proposta no artigo que serviu como base. Finalmente, o \textbf{Capítulo \ref{chap:cap7_conclusao}} apresenta consideração do presente projeto de pesquisa.