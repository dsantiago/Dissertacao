\chapter{Considerações}
\label{chap:conclusao}

Este trabalho, executado mediante uma metodologia de avaliação da literatura, foi decidido aplicar técnicas utilizadas na avaliação de câncer pulmonar, para o escopo de CH. Foi utilizada uma base de dados pública com exames de diversos pacientes, incluindo imagens da fase de diástole a qual foi utilizada para treinar o modelo. Os dados unem informações radiômicas de textura, sendo estar as \textit{features} de primeira ordem e \gls{GLCM}. Como \textit{features} radiômicas, foi utilizado um dos modelos mais robustos usados até dias contemporâneos, a \textit{ResNet}. A intuição é a de que a unificação informações de escopos diferentes só tendem a enriquecer e trazer resultados melhores do que em um cenário isolado.

A arquitetura proposta também deve ser aplicada e avaliado no conjunto de dados da \gls{InCor} o que ajudará a trazer mais percepções e ideias. Os resultados, por hora pouco satisfatórios, em nada desmerecem o esforço pois é possível notar que há bastante abertura e possibilidade de ideias propositivas a cerca deste tema. 