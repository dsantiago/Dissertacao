\chapter{CONCLUSÃO}
\label{chap:cap7_conclusao}

Este trabalho apresentou uma abordagem inovadora para a classificação de cardiomiopatias utilizando imagens de ressonância magnética cardiovascular (RMC) combinadas com mecanismos de atenção, promovendo a integração entre características radiômicas e profundas. A proposta foi construída com base em fundamentos teóricos sólidos, utilizando técnicas modernas de aprendizado profundo, como a arquitetura \gls{SE} Net e mecanismos de autoatenção, para explorar características discriminantes presentes em imagens médicas.

Os resultados obtidos destacaram a eficiência do modelo proposto na identificação de padrões associados a cardiomiopatias, superando abordagens tradicionais em termos de acurácia e generalização. A fusão de informações radiômicas, que capturam detalhes texturais e estatísticos, com características profundas extraídas de redes convolucionais, mostrou-se eficaz para lidar com a complexidade intrínseca das cardiomiopatias hipertróficas (CMH) e dilatadas (CMD). Além disso, o uso de mecanismos de atenção também permitiu priorizar as regiões mais relevantes das imagens, promovendo maior expressividade do modelo.

A utilização do mecanismo SE-ResNet e da atenção espacial no domínio de imagens médicas representou uma abordagem inovadora ao priorizar informações relevantes no mapa de características. Essa técnica não apenas melhorou o desempenho do modelo, mas também ofereceu uma forma mais intuitiva de explicar as decisões tomadas pela IA, um aspecto crucial para sua aceitação na prática clínica.

Embora os resultados sejam promissores, este trabalho também expôs limitações, como a necessidade de bases de dados mais robustas e diversificadas, além da dificuldade em interpretar os modelos de aprendizado profundo, um desafio recorrente na área médica. Contudo, o modelo proposto representa um avanço significativo na aplicação de inteligência artificial em diagnósticos médicos, contribuindo para o desenvolvimento de ferramentas mais precisas e acessíveis para o suporte à decisão clínica, especialmente no diagnóstico precoce e acompanhamento de cardiomiopatias. Essas ferramentas podem melhorar a eficiência dos profissionais de saúde, reduzindo o tempo de análise manual e permitindo intervenções mais rápidas e precisas.

O desenvolvimento e avaliação do modelo proposto trazem contribuições relevantes para a literatura de aprendizado profundo aplicado à medicina, ao demonstrar como características radiômicas e mecanismos de atenção podem ser combinados de forma eficaz. 

Por fim, os achados desta pesquisa abrem caminhos para futuras investigações, como a aplicação de mecanismos de atenção em outras doenças cardíacas e a integração de dados multimodais, incluindo genômicos e clínicos, para uma abordagem ainda mais abrangente e personalizada na medicina. O uso de técnicas avançadas, como aquelas desenvolvidas neste estudo, reforça o papel da inteligência artificial como uma aliada indispensável na medicina de precisão.

Como trabalhos futuros, sugere-se explorar a combinação de imagens médicas com dados clínicos e genéticos para criar modelos multimodais, que possam fornecer diagnósticos mais holísticos e personalizados. Além disso, o uso de técnicas como aprendizado federado pode ser investigado para proteger a privacidade dos dados enquanto se colabora em larga escala entre instituições.

%--------------------------------------------------------
% \section{PUBLICAÇÕES GERADAS} 
% \label{sec:cap7_publicacoes}

% \lipsum[1-1]