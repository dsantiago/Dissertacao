\chapter{CONCLUSÃO}
\label{chap:consideracoes_parciais}

Este trabalho foi executado mediante uma metodologia de avaliação da literatura e decidido aplicar técnicas radiômicas e técnicas de \gls{AP} no escopo de cardiomiopatia. Foi utilizada uma base de dados pública com exames de diversos pacientes, incluindo imagens da fase de diástole os quais foram utilizados para treinar o modelo. Os dados unem informações radiômicas de textura, dada a intuição de que a unificação de informações de diferentes escopos só tendem a enriquecer as informações e trazer maneiras melhores em como o modelo otimiza, obtendo resultados melhores do que em um cenário isolado.

A arquitetura proposta também deve ser aplicada e avaliada futuramente no conjunto de dados da \gls{InCor} o que ajudará a trazer mais percepções e ideias. Os resultados, pesar de pouco expressivos, indicam possibilidades de melhorias, principalmente em relação a seleção de características para fusão, quantidade de módulos de autoatenção, incorporação de camadas densas para seleção de características em conjunto com a concatenação acopladas ao modelo e obtenção de um conjunto maior de dados. 