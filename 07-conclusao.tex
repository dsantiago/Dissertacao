\chapter{Considerações}
\label{chap:conclusao}

Este trabalho foi executado mediante uma metodologia de avaliação da literatura e decidido aplicar técnicas radiômicas de de \gls{DL} no escopo de CH. Foi utilizada uma base de dados pública com exames de diversos pacientes, incluindo imagens da fase de diástole os quais foram utilizados para treinar o modelo. Os dados unem informações radiômicas de textura, dada a intuição de que a unificação de informações de diferentes escopos só tendem a enriquecer as informações e trazer maneiras melhores em como o modelo otimiza, obtendo resultados melhores do que em um cenário isolado.

A arquitetura proposta também deve ser aplicada e avaliada futuramente no conjunto de dados da \gls{InCor} o que ajudará a trazer mais percepções e ideias. Os resultados, por hora pouco expressivos, em nada desmerecem o esforço pois é possível notar que há bastante abertura e possibilidade de melhorias e ideias propositivas a cerca deste tema. 